% Chapter Template

\chapter{Conclusion} % Main chapter title

\label{Chapter5} % Change X to a consecutive number; for referencing this chapter elsewhere, use \ref{ChapterX}

Applications of Machine Learning in education are ubiquitous, from learning analytics to machine translation to summarization. One field of education that can greatly benefit from the applications of natural language processing is distance education. Teaching foreign languages by means of distance learning requires tremendous efforts from instructors and course developers to prepare online materials, learning activities, and resources. This report examines the applicability of a supervised classification technique called logistic regression in two educational scenarios: detection of grammatical difficulty and automatic reading comprehension. The simplicity and practicality of the proposed solutions to each of these problems make them uniquely useful in onsite and online teaching environments. 

As we pointed out in Chapter 2, representing a written text is a crucial step in the machine learning pipeline, and to do so, we have explored some of the most common methods such as Bag-of-Words, TF-IDF, and word embeddings.  We have also identified the advantages and disadvantages of each method in addition to techniques of dimensionality reduction such as clustering, latent semantic allocation, and stemming and lemmatization. Moreover, we have explained the workflow of classification as a supervised machine learning, starting from feature extraction through classification to evaluation of performance. 

In Chapter 3, we examined the use of multinomial logistic regression classifier to classify a written text according to how difficult a foreign language learner of English sees it using the educational standards of the Common European Framework Reference (CEFR). We also showed how a logistic regression classifier could fit within a semi-supervised framework such as Bootstrapping to work on low resources data. 

In Chapter 4, we investigated another use for logistic regression classification in method on a novel educational task, automatic reading comprehension. We have demonstrated that using logistic regression on two stages can compete with relatively advanced and complex neural network architecture commonly used for this purpose. The main catalyst for this success comes from using a set of linguistically rich features such as constituency parser and sentential embeddings. The performance of the model, the analysis of error made, as well as the simplicity and efficiency of the model make it feasible to be applied to onsite or virtual learning environments.


\section{Summary of Contributions}

The research offers three main contributions. First, to the best of our knowledge, we are the first to propose an automated solution to classify a written English text by following pedagogic standards (CEFR). Second, we have \emph{cautiously} applied a bootstrapping technique to supplement the training data from unlabeled corpora, which led to 10\% F1 increase in performance. Finally, we created a novel, two-stage system based on multinomial logistic regression that not only answers comprehension questions given reading passages but it also abstains from answering if a question is unanswerable within the reading passage. 

%-----------------------------------
%    SUBSECTION 1
%-----------------------------------
\section{Recommendations and Future Work}

We have seen that each of the tasks we tackled in this paper can help solve a particular educational scenario a foreign language instructor may encounter in a face-to-face or online learning settings. However, there still exist some challenges that need the attention of NLP researchers. The solution proposed in Chapter 4 finds answers to comprehension questions, but it still requires the questions as input. This means that teachers and material developers must create the questions. Thus, future studies should investigate the process of automatically creating different types of comprehension questions given reading passages. Such a system, complemented by the system we proposed, can make an automatic tool of learning activity creation that is ready to be deployed in Learning Management Systems (LMS) or Courseware Platforms. As for improving the performance of the current automatic reading comprehension system, we recommend incorporating more linguistic features such as information derived from Named Entity Recognition, Dependency Parsing, and/or Semantic Role Labeling. 

Due to the lack of training examples on each of the six classes in the grammatical detection task, we merged every two classes into one super-class to be the output. However, to provide more specificity to error detection, we recommend applying more iterations of cautious bootstrapping to obtain more training examples on each of the six classes (A1, A2, B1, B2, C1, and C2). More training examples can also reduce the state of data imbalance experienced in this task, and thereby eliminating the need for adjusting the class weights. The features we used are based on Bag-of-Words and TF-IDF, and we recommend using word embeddings or the sentence embeddings technique we have used for automatic reading comprehension task in Chapter 4. Another recommendation is to use easy-to-interpret classification algorithms like decision trees. This brings the advantage of providing "if-then" classification rules that humans can read and understand. Furthermore, as the task involves grammatical structures, we recommend designing features using constituency and dependency parsers. Finally, we encourage future studies to investigate how close will be the rules generated from the last two recommendations to the ones CEFR experts devised. 


