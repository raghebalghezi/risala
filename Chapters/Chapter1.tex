% Chapter Template

\chapter{Text Representation} % Main chapter title

\label{Chapter1} % Change X to a consecutive number; for referencing this chapter elsewhere, use \ref{ChapterX}

%----------------------------------------------------------------------------------------
%	SECTION 1
%----------------------------------------------------------------------------------------

\section{Introduction} 

Technically, a text is a set of symbols saved in a digital format. Some
of these formats are intended only for the distribution of documents
such as Portable Document Format (PDF) or PS (PostScript Format), while
others are determined by environments with well-defined rules such as
the Microsoft Word DOC format, the Open Office SXW format, or the LATEX
format. The goal is to unify the representation of textual documents
whatever their formats by transforming them into a set of terms
(features) that can be easily used by learning algorithms.

In order to apply learning algorithms on text, it is necessary to create
a numerical representation of the text and assign a weight to each
feature in text. This weighting is extremely important as it will
influence the results of learning algorithms. Thus, a second objective
is to improve weight the features.

The number of features is a very important factor on which the
performance of Text Classification depends. Indeed, several learning
algorithms are unable to handle a large number of features. For this
purpose, it is necessary to reduce the dimensionality of the
representation space by keeping only the best features. Reduction
methods are used to select or extract the most important features. In
this chapter, we will present the different methods of text
representation, specifying their advantages and disadvantages, and then
mention the weighting techniques most used in Text Classification as
well as the techniques used to reduce dimensionality.

\section{Classification as a Machine Learning Task}
Classification is a supervised Machine learning task aims to select a correct class for a given input, or more generally a task of “ assigning objects from a universe to two or more classes or categories ” \citep{manning1999foundations}. In natural language processing, the tasks of tagging, word sense disambiguation and spam filtering are all examples of supervised machine learning classification. A classifier is an algorithm that quantitatively models the relationship between a set of inputs and their associated output such that it generalizes well to new input data. In the task of spam filtering, for instance, input data are set of sentences with binary labels (SPAM, Not SPAM). 

\textbf{CHECKMEOUT}: Supervised Classification has two main phases: Training and Prediction. During training, a feature extractor is used to convert each input value to a feature set. These feature sets, which capture the basic information about each input that should be used to classify it, are discussed in the next section. Pairs of feature sets and labels are fed into the machine learning algorithm to generate a model. In prediction phase, the same feature extractor is used to convert unseen inputs to feature sets. These feature sets are then fed into the model, which generates predicted labels.

\begin{figure}
\includegraphics[scale=.8]{../Figures/mlpipeline.png} \centering
\caption{A figure shows the machine learning workflow}
\end{figure}

 \section{Logistic Regression}
 Logistic regression is one of the most common machine learning classifier in natural language processing tasks. It is also considered the building blocks of neural network models. Logistic regression belongs to a group of probabilistic classifiers called discriminative models which, unlike its generative counterparts like Naive Bayes classifiers, is to try to learn to distinguish the classes. More formally, given a document $d$ and a class $c$ , logistic regressions attempts to compute the conditional probability $P(c|d)$. 
 According to \citep{jurafsky2014speech}, classification using logistic regression, like any other supervised classification algorithm, has four main components: 

\begin{list}{•}{}
 \item Feature Representation: a method of numerically represent language data (characters, words and sentences, etc) input $x$ as feature vector $[x_{1},x_{2},x_{3},...x_{n}]$ where $i$  represents an observation in the input data.

\item Classification function: computes an estimation to class $\hat{y}$ via $p(y|x)$. In the case of binary logistic regression, the sigmoid function is used, while a softmax is used for multinomial classification. 
 
\item An objective function involving minimizing error on training examples (cross entropy)

\item An optimizer to help find the minimum of objective function 
 (stochastic gradient descent).
 \end{list}

Logistic regression also has two phases: training and testing. In training phase, the goal is to find the optimal weights that best account for mapping input features $x_{i}$ to the labelled output $y_{i}$ of a subset of the data. In testing, we measure the generalizability of weights learned in the training phase on the remaining set of the data.

Given a dataset $\mathcal { D } = \left\{ \left( \boldsymbol { x } ^ { ( i ) } , y ^ { ( i ) } \right) \right\} _ { i = 1 } ^ { N }$, the weights  are estimated by maximum conditional likelihood, 

\begin{equation}
\log \mathrm { p } \left( \boldsymbol { y } ^ { ( 1 : N ) } | \boldsymbol { x } ^ { ( 1 : N ) } ; \boldsymbol { \theta } \right) = \sum _ { i = 1 } ^ { N } \log \mathrm { p } \left( y ^ { ( i ) } | \boldsymbol { x } ^ { ( i ) } ; \boldsymbol { \theta } \right)
\end{equation}
\begin{equation}
\sum _ { i = 1 } ^ { N } \boldsymbol { \theta } \cdot \boldsymbol { f } \left( \boldsymbol { x } ^ { ( i ) } , y ^ { ( i ) } \right) - \log \sum _ { y ^ { \prime } \in \mathcal { Y } } \exp \left( \boldsymbol { \theta } \cdot \boldsymbol { f } \left( \boldsymbol { x } ^ { ( i ) } , y ^ { \prime } \right) \right)
\end{equation}





\section{Feature Representation}

A feature is defined as any element that can be used as an attribute in classification algorithms. Choosing what features to use to represent a text or a document not only has a major impact on the entire classification process, but it is also arguably the most difficult phase in any natural language processing task. The reasons go to several fundamental questions on how to represent language computationally. For example, on what level should we represent a text: \emph{character-wise}, \emph{word-wise} or \emph{sentence-wise}, or even more specifically morphologically, etymologically or phonologically? Do we treat a word e.g. \emph{try} and all its morphological variations e.g. \emph{tries}, \emph{trying}, \emph{tried} similarly or differently? How to represent and model the semantics of human language computationally? Questions like these and many others are the core of natural language processing; therefore, we briefly explore some of the most common text representation methods.

\subsection{Frequency-based Methods}

\subsubsection{Bag-of-Words Representation}

The "bag of words" representation is the simplest and most intuitive
representation. It represents each document by a vector
whose component corresponds to the number of occurrences of a word in
the document.  Words have the advantage of having an
explicit meaning. Nevertheless, the implementation of this
representation raises several difficulties. The first difficulty is that
of the delimitation of words in a text. Indeed, we still can not define
what is a word. A word as being a sequence of
characters belonging to a dictionary, or formally, as being a sequence
of characters separated by spaces or punctuation characters. This
definition is not valid for all languages. Indeed, languages such as
Chinese or Japanese do not separate their words by spaces. Add to this
that some separators can be part of certain words (for example: today,
127.0.0.1, potato, etc.). Another difficulty concerns the management of
compound words (for example: rainbow, potato, etc.) and acronyms (like:
IBM, CAF, CAN, etc.). Consideration of these cases requires quite
complex language treatments. This representation of texts excludes any
grammatical analysis and any notion of order between words and therefore
semantically distant texts can have the same representation. For
example, the sentences \texttt{the\ wolf\ eats\ the\ goat} and
\texttt{the\ goat\ eats\ the\ wolf} is given the same representation
despite being semantically different.

% \subsubsection{Representation by Sentences}

Bag-of-words representation excludes any notion of order and
relationship between the words of a text, several studies have
attempted to use sentences as features instead of words \citep{fuhr1991probabilistic} \citep{tzeras1993automatic}.
The use of the sentences makes it possible to solve the problem of
ambiguity generated by the use of the representation "bag of words". For
example, the word \texttt{mouse} has several possible meanings while
optical mouse and domestic mouse has no ambiguity. Although the
sentences have the advantage of better keeping the semantics in relation
to the words, their use as features did not lead to the expected
results. According to Lewis \citep{lewis1992representation}, this representation is penalized
by the large number of possible combinations which leads to low and too
random frequencies. One solution proposed in \citep{caropreso2001learner} was to consider a
sentence as a set of contiguous (but not necessarily ordered) words that
appear together but do not necessarily respect grammatical rules.

% \subsubsection{Representation by lemmas or lexical roots}

A modification to the "bag of words" is representation by lemmas or lexical roots which replaces  each word by its canonical form so that words with different
forms (singular, plural, masculine, feminine, present, past,
future, etc.) can have in a single form called a canonical form.
Grouping the different forms of a word offers the advantage of  reducing the dimensionality of the learning space. Indeed, in the representation "bag-of-words", each form of a word is given a dimension; while with lemma representation the different forms of one word will be merged into one dimension. For example, words such as play, player, players, play, play, play, cheek, etc. will be replaced by a single describer ie the root played or the lemma play.


Lemmatization and stemming are the two techniques used to find the canonical form of a word. Lemmatization uses a knowledge base containing the different inflected forms corresponding to the different possible lemmas. Thus, the inflected forms of a noun will be replaced by the singular masculine form while the different inflected forms of a verb will be replaced by the infinitive form. Lemmatization requires the use of a dictionary of inflected forms of language as well as a grammar labeler. An efficient algorithm, named TreeTagger \citep{schmid1994probabilistic}, has been developed for thirteen different languages: German, English, French, Italian, Dutch, Spanish, Bulgarian, Russian, Greek, French Portuguese, Chinese, Swahili and Old French. Stemming uses a knowledge base of syntactic and morphological rules and to transform words into their roots. One of the most well-known stemming algorithms for the English language is Porter's algorithm \citep{porter1980algorithm}. Lemmatization is more complicated to implement since it depends on the grammatical labellers. In addition, it is more sensitive to misspellings than stemming.

\begin{lstlisting}
def stem(word):
    for suffix in ["ing","ly","ed","ious","ies","ive","es","s","ment"]:
        if word.endswith(suffix):
            return word[:-len(suffix)]
    return word
\end{lstlisting}

% \subsubsection{Representation by n-grams}

A slightly better approach to represent text that makes use of context is the representation by n-grams. It consists of breaking the text into moving sequences of $n$ consecutive tokens. For example, the trigram segmentation of the word "language" gives the following 3-grams: lan, ang, ngu, gua, uag, age. The notion of n-grams was introduced by  \citep{shannon1948mathematical}; he was interested in predicting the appearance of certain characters according to the other characters. Since then, n-grams have been used in many areas such as speech recognition, documentary research, and so on. The use of n-grams offers the following advantages: 1) language-independent 2) Requiring no prior segmentation of the document. 3) Less sensitive to misspellings. 4) effective method to automatic language identification.


% \subsubsection{Representation by concepts}

Depending on the task, we may sometimes need to aggregate words with certain linguistic relation in concepts. For example, we replace co-hyponyms \emph{dog} and \emph{wolf} with their hypernym \emph{canine}; or \emph{vehicle} and \emph{car} with \emph{automobile}. Concepts, defined as units of knowledge, can be used as features for the purpose of solving the ambiguity problem as well as the problem of synonymy. Indeed, each concept represents a unique meaning that can be expressed by several synonymous words. Similarly, a word with many meanings (sense) is found mapped to several concepts. Thus, a document containing the word \emph{vehicle} may be indexed by other words such as car or automobile. The transition from a word representation to a concept representation requires the use of semantic resources external to the content of documents such as: semantic networks, thesauri and ontologies. As a result, the performance of such a representation crucially depends on the semantic richness of the resources used in terms of the number of concepts and relationships between these concepts.

\subsubsection{Weighting of the features}

Co-occurrence methods such as bag-of-words has major drawbacks, one of which is term with higher frequency are more important than those with less frequency. To address this problem, several normalization methods were developed to remove the bias towards more frequent terms. One of these method is Term Frequency Inverse Document Frequency (TFIDF).

TFIDF has two main parts. Term Frequency is proportional to the frequency of the term in the document (local weighting). Thus, the longer the term is common in the document, the more important it is. It can be used as is or in several variations \citep{singhal1997learning} \citep{sable2001using}.

$$t f _ { i j } = f \left( t _ { i } , d _ { j } \right)$$

$$t f _ { i j } = 1 + \log \left( f \left( t _ { i } , d _ { j } \right) \right)$$

$$ t f _ { i j } = 0.5 + 0.5 \frac { f \left( t _ { i } , d _ { j } \right) } { m a x _ { t _ { i } \in d _ { j } } f \left( t _ { i } , d _ { j } \right) }$$

where $f(t_{i},d_{j})$ is the term frequency of document $j$

Inverse Document Frequency measures the importance of a term throughout the collection (overall weighting). A term that often appears in the document base should not have the same impact as a less frequent term. Indeed, the terms that appear in the majority of documents do not have any discriminating power to distinguish documents from each other and must therefore have low weightings. The IDF weighting is inversely proportional to the number of documents containing the term to be weighted. Thus, the more the term appears in several documents the less it is discriminating and is assigned a low weighting. The IDF weighting is generally expressed as follows:

$$i d f \left( t _ { i } \right) = \log \left( \frac { N } { d f \left( t _ { i } \right) } \right)$$

where $df(t_{i})$ is the term frequency of feature $i$ and $N$ is the number of documents in the corpus.

The TFIDF weighting combines the two weightings TF and IDF in order to
provide a better approximation of the importance of a term in a
document. According to this weighting, for a term to be important in a
document, it must appear frequently in the document and rarely in other
documents. This weighting is given by the product of the local weighting
of the term in the document by its overall weighting in all the
documents of the corpus.

$$t f i d f \left( t _ { i } , d _ { j } \right) = t f _ { i j } \times \log \left( \frac { N } { d f \left( t _ { i } \right) } \right)$$


TFC is a modification to TFIDF that overcomes the major drawback of the
TFIDF measure, namely that the length of the documents is not taken into
consideration by adding a normalization factor. The TFC weighting of a
term $i$ in a document $j$ is calculated as follows:

$$T F C _ { i j } = \frac { T F I D F \left( t _ { i } , d _ { j } \right) } { \sqrt { \sum _ { k = 1 } ^ { T } T F I D F \left( t _ { k } , d _ { j } \right) ^ { 2 } } }$$

\subsection{Dimensionality reduction}

The text representation method explored so far, despite its usefulness in a variety of tasks in natural language processing, share sparsity as one main problem. The number of features generated using these methods can easily exceed the tens of thousands, and not only does negatively influence the categorization process, it is also very computationally expensive in terms of hardware resources. In addition, the higher the dimensions of the features are, the weaker the features become. This problem is known as the \emph{curse of dimensionality} \citep{bellman2015adaptive}. To remedy this issue,  techniques, borrowed from the field of information theory and linear algebra, have been developed to reduce the feature space dimensionality without losing a lot of information. 

% Dimensionality reduction techniques can be done using probabilistic and linear algebraic techniques such as principal component analysis (PCA) and linear discriminant analysis (LDA) that aim to project highly dimensional data into lower dimension space. 
% \paragraph{Extraction of terms}

Clustering \citep{baker1998distributional} \citep{sable2001using} \citep{slonim2001power} can be used to reduce dimensionality. It consists of representing the documents in a new representation space other than the original one. Each dimension of the new representation space groups terms that share the same meaning. Thus, the documents will no longer be represented by terms but rather by groupings of terms representing semantic concepts. This new space of representation offers the advantage of managing the synonymy since the synonymous terms will appear in the same groupings. Likewise, the fact that a term can be included in several groupings also makes it possible to manage the polysemy. Another very interesting method to reduce dimensionality, proposed by \citep{deerwester1990indexing} is Latent Semantic Allocation (LSA). It uses singular value decomposition of the document $x$ term matrix to change the representation by keeping only the $k$ axes of strongest singular values. However, LSA is very expensive in terms of calculation time during learning. Likewise, with each new document, it is necessary to redo the whole grouping process.

\subsection{Distributional Method}
The methods we have discussed so far are based mainly on the idea of frequency. Another main approach for text representation and natural language of processing is the distributional similarity. The basic idea of the distribution of similarity is to represent the meaning of a word using words which co-occurs in the same context. "You shall know a word by the company it keeps" \citep{firth1957synopsis}. For example, the words 'dog' and 'cat' have the same meaning in the following two sentences: I have a cat, I have a dog. While this idea might not be adequate to capture or account for the complexity or the sophistication of human language, it has succeeded tremendously in a variety of lexicalized NLP tasks. In fact, it has become the de facto representation method of text. The need for such method does not only come from the need for more accurate numerical representation, but its density compared to other methods makes it less computationally expensive. Word2Vec \citep{mikolov2013distributed} and GloVe \citep{pennington2014glove} are two famous algorithms for creating word embedding. 

The intuition of word2vec is to train a classifier on a binary prediction task: “Is
word $w$ likely to show up near X?”, where $X$ is the word to which we want to find embeddings. Then, we use the weights learned by the classifier as embeddings. Skip-gram algorithm \citep{mikolov2013distributed} , first, initializes embeddings randomly. Then, it iteratively updates the embeddings of each word $w$ to be equal to the words they occur with in the same context. Finally, it injects $k$ number non-neighbor words as negative examples. 

\section{Classification and Objective Function}
\subsection{Classification Functions}
What we have discussed so far gives us a method to numerically translate language from a form that is uniquely comprehensible to humans into a representation that a computer can understand, and can somehow capture the characteristics of human language. However, in order for the logistic regression to to learn to classify, a classification function should be utilized. Depending on whether the classification task is binary or multinomial, there are two main functions used: sigmoid and softmax. Sigmoid function, used for binary classification, outputs 1 if an observation input (feature vector) is a member of a particular class e.g. (spam), and 0 otherwise. In other words, it calculates $P(y=1|x)$ for spam text and $P(y=0|x)$ for non-spam text. The mathematical form of the Sigmoid function or (logistic function, hence comes the name of the classifier) is as follow:

$$Sig ( x ) = \frac { 1 } { 1 + e ^ { - (w \cdot x +b)} }$$ 

where $x$ represent the input vector, $w$ is the learned weights and $b$ is the bias term or the intercept of the linear equation. One mathematical property of the Sigmoid function is that it maps any input between 0 and 1, and to make it as probability we need to make the sum of $P(y=1)$ and $P(y=0)$ equals to 1. 

$$ P ( y = 1 ) = \sigma ( w \cdot x + b ) $$
$$ P ( y = 0 ) = 1 - \sigma ( w \cdot x + b ) $$

Now to make decision, the classifier declares an observation input as \emph{Yes} (if it is a member of a class spam) if the probability is greater than certain threshold say 0.5, and declares an observation input as \emph{No} otherwise. 

$$  \hat { y } = \left\{ \begin{array} { l l } { 1 } & { \text { if } P ( y = 1 | x ) > 0.5 } \\ { 0 } & { \text { otherwise } } \end{array} \right.  $$


In the case of multinomial classification, the logic remains the same expect for the classification function where Softmax is used rather than Sigmoid. Softmax 

$$p ( y = i | x ) = \frac { e ^ { w _ { i } \cdot x + b _ { i } } } { \sum _ { j = 1 } ^ { k } e ^ { w _ { j } \cdot x + b _ { j } } }$$

works on normalizing the prediction of an observation by the values of all other classes in order to produce valid probability distribution. 

\subsection{Objective Functions}

During the learning process, we need to answer the question of \emph{how correct is the estimated output $\hat{y}$ of classifier from the true output $y$?}, and thus our objective \emph{function} is to minimize the difference between the estimated output and the true one such that the weights learned during this minimization process are, hopefully,  generalizable enough to correctly label observations unseen in the training phase. So we can imagine an objective function to be a difference between the $\hat{y}$ and $y$ i.e. $\left | \hat{y} - y \right |$, but for mathematical convenience we need a non-convex objective function (or also commonly referred to as loss function)  that can be easily optimized. Thus, we use maximum likelihood estimation. 

Binary classification can be seen as Bernoulli distribution since the outcome is either 0 or 1. More formally, the probability of predicting class $y$ given input $x$ is  $p ( y | x ) = \hat { y } ^ { y } ( 1 - \hat { y } ) ^ { 1 - y }$. Minimizing the probability is the same as maximizing the negative log likelihood:
 $$L _ { C E } ( \hat { y } , y ) = - \log p ( y | x ) = - [ y \log \hat { y } + ( 1 - y ) \log ( 1 - \hat { y } ) ]$$

Finally, by plugging in the value of $\hat{y}$ we get:
$$ L _ { C E } ( w , b ) = - \frac { 1 } { m } \sum _ { i = 1 } ^ { m } y ^ { ( i ) } \log \sigma \left( w \cdot x ^ { ( i ) } + b \right) + \left( 1 - y ^ { ( i ) } \right) \log \left( 1 - \sigma \left( w \cdot x ^ { ( i ) } + b \right) \right) $$, where $m$ is the number of training examples.

On the other hand, multinomial classification uses the same process except that it uses a different classification function \emph{softmax} and hence has different (multinomial) distribution. So the loss function for a single example $x$ is the sum of the logs of the $K$ output classes:
$$ L _ { C E } ( \hat { y } , y ) = - \sum _ { k = 1 } ^ { K } 1 \{ y = k \} \log \frac { e ^ { w _ { k } \cdot x + b _ { k } } } { \sum _ { j = 1 } ^ { K } e ^ { w _ { j } \cdot x + b _ { j } } } $$

\section{Optimization}

The objective functions derived in the previous section is optimized  using numerical methods. Several algorithms are used for this purpose such as Stochastic Gradient Descent, that uses first derivative (gradient) information to find a minimum of a function; Newton-Raphson algorithm uses Second derivative information to find a minimum of a function. Discussing the details of how these algorithms work and the mathematical derivation of gradients is beyond the scope of this work. Interested readers are referred to \citep{jurafsky2014speech} for more details on the derivation of Stochastic Gradient Descent. In addition, current machine learning frameworks like Google's Tensorflow and Facebook's Pytorch offer automatic differentiation techniques to find an optimum of a function. 

\section{Evaluation Metrics}
\subsection{Confusion Matrix}

A confusion matrix is a method to visualize the results of a classification algorithm. For the binary classification, the algorithm can be used to predict whether a test sample is either 0, or 1. As a way to measure how well the algorithm performs, we can count four different metrics, here 1 defined as positive and 0 defined as negative:

\begin{enumerate}

\item True positive (TP), the algorithm classifies 1 where the correct class is 1.
\item False positive (FP), the algorithm classifies 1 where the correct class is 0.
\item True negative (TN), the algorithm classifies 0 where the correct class is 0.
\item False negative (FN), the algorithm classifies 0 where the correct class is 1.

\end{enumerate}

\begin{figure}[hbtp]
\caption{Confusion Mtrix Example}
\includegraphics[scale=.5]{../Figures/Confuse_Mat_Example.png}\centering
\end{figure}

\subsection{Precision, Recall and F-Score}

Precision and Recall are two popular measurement to evaluate the performance of supervised classification methods. Precision is defined as:

$$\text { Precision } = \frac { True Postive } { True Postive + False Positive }$$

, recall defined as:

$$\text { Recall } = \frac { True Postive } { True Postive + False Negative }$$

and F-score is defined as the harmonic mean of both precision and recall:

$$F _ { 1 } = 2 \cdot \frac { \text { precision } \cdot \text { recall } } { \text { precision } + \text { recall } }$$

A classifier with high precision means it classifies almost no inputs as positive unless they are positive. A classifier high recall, on the other hand, would mean that the it misses almost no positive values. 
\section{Conclusion}

In this chapter, we briefly introduced some fundamental topics of Machine Learning theory required for natural language processing supervised classification tasks. We discussed two commonly used numerical methods to represent text computationally and their variants, Frequency-based methods: Bag-of word models (BoW) and Term Frequency - Inverse Document Frequency (tfidf), and Distributional similarity methods.In addition, we reviewed some techniques to reduce the dimensionality of feature space to reduce the complexity of the training and save computational resources. Finally, we covered how logistic regression as a probabilistic algorithm can be used for binary or multinomial classification along with the evaluation metrics used to  gauge its performance.